\section{Problem Statement and Objectives}

\subsection{Problem Statement}
The current hands-on form development, while concrete and concise, is rigid and hard to adapt across projects.
Due to the requirement of the ability to change the appearance of the forms for each project user, the current approach
leads to many problems with the maintainability of the code.

Aside from the maintainability, the current approach also suffers from lots of redundancy in the code.
Forms share many common elements, such as labels, input fields, and buttons, yet the current approach duplicates code
rather than reusing components.

\subsection{Objectives}

With the problems clearly defined, the goal of this thesis is to analyze the feasibility of a new approach to form
creation, one that successfully addresses the problems of the current approach.
More specifically, the goals are:

\begin{itemize}
    \item To analyze how much boilerplate and duplication exist in the current form creation approach.
    \item To design a schema-driven approach that unifies the form creation logic and supports custom appearance for each
    project.
    \item To implement a prototype of the new approach and evaluate developer productivity and maintainability.
    \item To recommend whether the new approach is feasible for production use and the best approaches for going forward.
\end{itemize}