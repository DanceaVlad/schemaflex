\section{Problem Statement and Objectives}

\subsection{Problem Statement}
The current hands-on form development, while concrete and concise, is rigid and hard to adapt across projects.
Because each project user may require custom styling, the current approach leads to many problems with the maintainability
of the code.
This is because each version of the form must be kept in sync with the others: when one core change is made, all current
versions of the form must be updated.

Aside from the maintainability, the current approach also suffers from lots of redundancy in the code.
Forms share many common elements, such as labels, input fields, and buttons, yet the current approach duplicates code
rather than reusing components.

\subsection{Objectives}

With the problems clearly defined, the goal of this thesis is to analyze the feasibility of a new approach to form
creation, one that successfully addresses the problems of the current approach.
More specifically, the goals are:

\begin{itemize}
\item \textbf{Quantify the current boilerplate.} Analyze representative forms in the current codebase.
\item \textbf{Design a unified schema approach.} Define a schema-driven approach that unifies the form creation logic and
    supports custom appearance for each project.
\item \textbf{Implement a prototype.} Build and test an end-to-end JsonForms proof-of-concept, then measure dev effort and maintainability.
\item \textbf{Recommend feasibility.} Evaluate production readiness and outline next steps for adoption.
\end{itemize}